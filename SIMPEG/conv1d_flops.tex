\documentclass[11pt,a4paper]{article}
\usepackage[utf8]{inputenc}
\usepackage{amsmath}
\usepackage{amsfonts}
\usepackage{amssymb}
\usepackage{graphicx}
\usepackage{booktabs}
\usepackage{hyperref}
\usepackage{geometry}
\usepackage{xcolor}
\usepackage{tikz}
\usetikzlibrary{arrows.meta, positioning, shapes.geometric}

\geometry{margin=2.5cm}

\title{Computational Cost Analysis of 1D Convolutional Layers\\
\large FLOP Calculation for Conv1D Operations}
\author{TFE4580 Specialization Project}
\date{\today}

\begin{document}

\maketitle

\section{Introduction}

In deep learning, understanding the computational cost of neural network operations is crucial for:
\begin{itemize}
    \item Deployment on resource-constrained devices (embedded systems, edge devices)
    \item Estimating inference time and power consumption
    \item Comparing different network architectures
    \item Optimizing network design for real-time applications
\end{itemize}

For 1D Convolutional Neural Networks (CNNs) used in time series classification, the convolutional layers typically dominate the computational cost, often accounting for 90-95\% of total FLOPs. This document provides a thorough derivation of FLOP calculations for Conv1D layers.

\section{Definitions and Notation}

\subsection{Input Tensor}
The input to a Conv1D layer is a 3D tensor with shape:
\begin{equation}
    \mathbf{X} \in \mathbb{R}^{N \times L_{\text{in}} \times C_{\text{in}}}
\end{equation}
where:
\begin{itemize}
    \item $N$ = batch size (number of samples)
    \item $L_{\text{in}}$ = input sequence length (number of time steps)
    \item $C_{\text{in}}$ = number of input channels (features per time step)
\end{itemize}

\subsection{Convolutional Kernel}
The learnable weight tensor (kernel) has shape:
\begin{equation}
    \mathbf{W} \in \mathbb{R}^{K \times C_{\text{in}} \times C_{\text{out}}}
\end{equation}
where:
\begin{itemize}
    \item $K$ = kernel size (temporal receptive field)
    \item $C_{\text{in}}$ = number of input channels
    \item $C_{\text{out}}$ = number of output channels (filters)
\end{itemize}

\subsection{Output Tensor}
The output tensor has shape:
\begin{equation}
    \mathbf{Y} \in \mathbb{R}^{N \times L_{\text{out}} \times C_{\text{out}}}
\end{equation}
where $L_{\text{out}}$ depends on the padding mode and stride.

\section{Output Length Calculation}

The output sequence length $L_{\text{out}}$ is determined by:

\subsection{Valid Padding (No Padding)}
\begin{equation}
    L_{\text{out}} = \left\lfloor \frac{L_{\text{in}} - K}{s} \right\rfloor + 1
\end{equation}
where $s$ is the stride (typically $s=1$ for feature extraction).

\subsection{Same Padding (Zero Padding)}
With ``same'' padding, the input is padded with zeros such that:
\begin{equation}
    L_{\text{out}} = \left\lceil \frac{L_{\text{in}}}{s} \right\rceil
\end{equation}
For stride $s=1$:
\begin{equation}
    L_{\text{out}} = L_{\text{in}}
\end{equation}

\section{FLOP Calculation Derivation}

\subsection{Single Output Element}

To compute a single output element $y_{n,l,c_{\text{out}}}$, we perform the following operation:

\begin{equation}
    y_{n,l,c_{\text{out}}} = \sum_{k=0}^{K-1} \sum_{c=0}^{C_{\text{in}}-1} x_{n,l \cdot s + k,c} \cdot w_{k,c,c_{\text{out}}} + b_{c_{\text{out}}}
\end{equation}

where $b_{c_{\text{out}}}$ is the optional bias term.

\subsection{Multiply-Accumulate (MAC) Operations}

Each output element requires:
\begin{itemize}
    \item \textbf{Multiplications:} $K \times C_{\text{in}}$ (one per kernel weight)
    \item \textbf{Additions:} $K \times C_{\text{in}} - 1$ (summing the products)
    \item \textbf{Bias addition:} 1 (if bias is used)
\end{itemize}

\subsection{Definition of FLOPs}

In deep learning, a \textbf{FLOP} (Floating Point Operation) typically refers to a single floating-point arithmetic operation. However, conventions vary:

\begin{itemize}
    \item \textbf{Multiply-Add as 2 FLOPs:} Each multiply-accumulate (MAC) operation consists of one multiplication and one addition, counted as 2 FLOPs.
    \item \textbf{Multiply-Add as 1 FLOP:} Some sources count each MAC as a single FLOP (also called a ``fused multiply-add'' or FMA).
\end{itemize}

\textbf{In this document, we use the convention that 1 MAC = 2 FLOPs.}

\subsection{FLOPs per Output Element}

Using the 2-FLOP-per-MAC convention:

\begin{equation}
    \text{FLOPs}_{\text{element}} = 2 \times K \times C_{\text{in}}
\end{equation}

If bias is included:
\begin{equation}
    \text{FLOPs}_{\text{element}} = 2 \times K \times C_{\text{in}} + 1
\end{equation}

\subsection{Total FLOPs for Conv1D Layer}

The total number of output elements is:
\begin{equation}
    N_{\text{elements}} = L_{\text{out}} \times C_{\text{out}}
\end{equation}

Therefore, the total FLOPs for a Conv1D layer (per sample) is:

\begin{equation}
    \boxed{\text{FLOPs}_{\text{Conv1D}} = 2 \times K \times C_{\text{in}} \times C_{\text{out}} \times L_{\text{out}}}
\end{equation}

With bias:
\begin{equation}
    \boxed{\text{FLOPs}_{\text{Conv1D}} = 2 \times K \times C_{\text{in}} \times C_{\text{out}} \times L_{\text{out}} + C_{\text{out}} \times L_{\text{out}}}
\end{equation}

\section{Example Calculation: PiClassifier Architecture}

Consider the PiClassifier CNN with the following architecture (input length = 50):

\subsection{Layer Specifications}

\begin{table}[h]
\centering
\caption{Conv1D Layer Parameters in PiClassifier}
\label{tab:layers}
\begin{tabular}{@{}lcccccr@{}}
\toprule
Layer & $C_{\text{in}}$ & $C_{\text{out}}$ & $K$ & Padding & $L_{\text{in}}$ & $L_{\text{out}}$ \\
\midrule
conv1 & 1 & 16 & 7 & same & 50 & 50 \\
conv2 & 16 & 32 & 5 & same & 25 & 25 \\
conv3 & 32 & 64 & 3 & same & 12 & 12 \\
\bottomrule
\end{tabular}
\end{table}

\textit{Note: $L_{\text{in}}$ for conv2 and conv3 is reduced by MaxPooling1D(pool\_size=2) after the previous layer.}

\subsection{FLOP Calculation for Each Layer}

\subsubsection{Layer 1: conv1}
\begin{align}
    \text{FLOPs}_{\text{conv1}} &= 2 \times K \times C_{\text{in}} \times C_{\text{out}} \times L_{\text{out}} \\
    &= 2 \times 7 \times 1 \times 16 \times 50 \\
    &= 2 \times 5600 \\
    &= \mathbf{11,200} \text{ FLOPs}
\end{align}

With bias ($+16 \times 50 = 800$):
\begin{equation}
    \text{FLOPs}_{\text{conv1}} = 11,200 + 800 = \mathbf{12,000} \text{ FLOPs}
\end{equation}

\subsubsection{Layer 2: conv2}
\begin{align}
    \text{FLOPs}_{\text{conv2}} &= 2 \times 5 \times 16 \times 32 \times 25 \\
    &= 2 \times 64,000 \\
    &= \mathbf{128,000} \text{ FLOPs}
\end{align}

With bias ($+32 \times 25 = 800$):
\begin{equation}
    \text{FLOPs}_{\text{conv2}} = 128,000 + 800 = \mathbf{128,800} \text{ FLOPs}
\end{equation}

\subsubsection{Layer 3: conv3}
\begin{align}
    \text{FLOPs}_{\text{conv3}} &= 2 \times 3 \times 32 \times 64 \times 12 \\
    &= 2 \times 73,728 \\
    &= \mathbf{147,456} \text{ FLOPs}
\end{align}

With bias ($+64 \times 12 = 768$):
\begin{equation}
    \text{FLOPs}_{\text{conv3}} = 147,456 + 768 = \mathbf{148,224} \text{ FLOPs}
\end{equation}

\subsection{Total Convolutional FLOPs}

\begin{align}
    \text{FLOPs}_{\text{total conv}} &= \text{FLOPs}_{\text{conv1}} + \text{FLOPs}_{\text{conv2}} + \text{FLOPs}_{\text{conv3}} \\
    &= 12,000 + 128,800 + 148,224 \\
    &= \mathbf{289,024} \text{ FLOPs}
\end{align}

\subsection{Comparison with Total Network FLOPs}

The complete PiClassifier network (including BatchNorm, Pooling, Dense layers) has approximately 306,776 FLOPs per inference. The convolutional layers account for:

\begin{equation}
    \frac{289,024}{306,776} \approx \mathbf{94.2\%}
\end{equation}

This confirms that \textbf{convolutional layers dominate the computational cost}.

\section{Computational Complexity Analysis}

\subsection{Big-O Notation}

The computational complexity of a Conv1D layer is:
\begin{equation}
    O(K \times C_{\text{in}} \times C_{\text{out}} \times L_{\text{out}})
\end{equation}

This is linear in each of the four dimensions. Key observations:
\begin{itemize}
    \item \textbf{Kernel size $K$:} Linear impact. Doubling $K$ doubles FLOPs.
    \item \textbf{Input channels $C_{\text{in}}$:} Linear impact. Each input channel adds to the computation.
    \item \textbf{Output channels $C_{\text{out}}$:} Linear impact. Each filter requires a full convolution.
    \item \textbf{Sequence length $L_{\text{out}}$:} Linear impact. Longer sequences require more operations.
\end{itemize}

\subsection{Design Trade-offs}

\begin{table}[h]
\centering
\caption{Design Strategies to Reduce Conv1D FLOPs}
\label{tab:tradeoffs}
\begin{tabular}{@{}lll@{}}
\toprule
Strategy & FLOP Reduction & Trade-off \\
\midrule
Reduce kernel size $K$ & Linear & May miss longer temporal patterns \\
Reduce filters $C_{\text{out}}$ & Linear & Reduced feature capacity \\
Add pooling layers & Reduces $L_{\text{out}}$ & May lose fine temporal resolution \\
Use depthwise separable conv & $\approx C_{\text{in}}$x reduction & Slightly different features \\
Use strided convolution & Reduces $L_{\text{out}}$ by $s$ & Faster but coarser representation \\
\bottomrule
\end{tabular}
\end{table}

\section{Real-World Performance Estimates}

\subsection{Inference Time Estimation}

Modern hardware can achieve varying FLOP throughput:

\begin{table}[h]
\centering
\caption{Estimated Inference Time for PiClassifier (306,776 FLOPs)}
\label{tab:inference}
\begin{tabular}{@{}llr@{}}
\toprule
Platform & Approx. GFLOPS & Estimated Inference Time \\
\midrule
CPU (Intel i7, single core) & 10-50 & $\sim$6--31 $\mu$s \\
GPU (NVIDIA RTX 3080) & 29,770 & $\sim$0.01 $\mu$s \\
Raspberry Pi 4 & 1-5 & $\sim$61--307 $\mu$s \\
Microcontroller (ARM Cortex-M4) & 0.01-0.1 & $\sim$3--31 ms \\
\bottomrule
\end{tabular}
\end{table}

\textit{Note: Actual performance depends on memory bandwidth, batch size, and implementation efficiency.}

\subsection{Comparison with Literature}

\begin{table}[h]
\centering
\caption{FLOP Comparison with Literature Models}
\label{tab:comparison}
\begin{tabular}{@{}lrrr@{}}
\toprule
Model & Total FLOPs & Conv FLOPs & Parameters \\
\midrule
PiClassifier (ours) & 307K & 289K & 11.5K \\
Thomas2024 1D-CNN & $\sim$13M & $\sim$12.5M & $\sim$1.3M \\
Minhas2024 DL-MMD & $\sim$1.2M & $\sim$1.1M & $\sim$50K \\
\bottomrule
\end{tabular}
\end{table}

The PiClassifier achieves a \textbf{40x reduction in FLOPs} compared to Thomas2024, while maintaining competitive accuracy.

\section{Summary Formula}

For a Conv1D layer with:
\begin{itemize}
    \item Kernel size $K$
    \item Input channels $C_{\text{in}}$
    \item Output channels (filters) $C_{\text{out}}$
    \item Output length $L_{\text{out}}$
    \item Bias enabled
\end{itemize}

The total FLOPs is:

\begin{equation}
    \boxed{\text{FLOPs} = L_{\text{out}} \times C_{\text{out}} \times (2 \times K \times C_{\text{in}} + 1)}
\end{equation}

Or equivalently:
\begin{equation}
    \boxed{\text{FLOPs} = 2 \times K \times C_{\text{in}} \times C_{\text{out}} \times L_{\text{out}} + C_{\text{out}} \times L_{\text{out}}}
\end{equation}

\section{Conclusion}

Understanding the FLOP calculation for Conv1D layers is essential for:
\begin{enumerate}
    \item \textbf{Architecture design:} Choosing appropriate kernel sizes and filter counts
    \item \textbf{Deployment planning:} Estimating computational requirements for target hardware
    \item \textbf{Model comparison:} Objectively comparing network efficiency
    \item \textbf{Optimization:} Identifying bottlenecks for acceleration
\end{enumerate}

For the PiClassifier, the convolutional layers account for approximately 94\% of total FLOPs, confirming that \textbf{Conv1D analysis is sufficient for understanding the network's computational cost}.

\end{document}
